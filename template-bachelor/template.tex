\documentclass[UTF8,zihao=-4,twoside]{ctexart}
\usepackage{fancyhdr}%页眉页脚
\usepackage{graphicx}%图片
\usepackage{amsmath,amsfonts,amssymb}%数学公式,符号
\usepackage{array}%表格、数组
\usepackage{cases}%左半边括号
\usepackage{listings}%附录
%\usepackage{lastpage}
\usepackage{subfigure}
\usepackage{helvet}
\usepackage{setspace}
\usepackage{upgreek}
\usepackage{multirow}
\usepackage{appendix}
\usepackage{titlesec}
\usepackage{titletoc}
\usepackage{cite}
\usepackage{tabu}
\usepackage{booktabs}
\usepackage{bm}
\usepackage{xeCJK}
\usepackage{latexsym}
\usepackage{caption}
\usepackage{wrapfig}
\usepackage{fontspec}
%\usepackage{mathtime}
%\usepackage{subcaption}
\usepackage{float}
\usepackage{verbatim}
%\usepackage{mathspec}
\usepackage[T1]{fontenc}
\usepackage{mathptmx}
%\usepackage{newtxmath}
\usepackage{ifthen}
\usepackage{indentfirst}
\usepackage{zhnumber}
\usepackage{amsthm}
\usepackage[a4paper,left=2.5cm,right=2cm,top=2cm,bottom=2cm,headsep=4pt]{geometry}


\setCJKmainfont[AutoFakeBold=2.7,AutoFakeSlant=0.2]{SimSun}%BoldFont=SimHei,ItalicFont=KaiTi(\bf,\it)
\setCJKsansfont{SimHei}
\setCJKmonofont{FangSong}
\setmainfont{Times New Roman}
\setsansfont{Calibri}%无衬线字体\sf   Arial、Calibri、Latin Modern Sans
\setmonofont{Courier}%等宽字体/打印机字体\tt
%\setmathfont{Latin Modern Math}
\linespread{1.25}
%衬线字体\rm  Times New Roman、Cambria、Latin Modern
%倾斜\sl,数字斜体\mit,小体大写字母\sc
%\renewcommand\familydefault{\rmdefault}

%\setCJKfamilyfont{songFy}{SimSun}[AutoFakeBold=10]
%\newcommand*{\songFt}{\CJKfamily{songFy}}
\newCJKfontfamily{\songFt}{SimSun}[AutoFakeBold=4]
\newfontfamily{\sansFt}{Latin Modern Sans}[AutoFakeBold=4]
\newCJKfontfamily{\stxingkai}{STXingkai}[AutoFakeBold=2]

\newtheoremstyle{thmstyle}{0pt}{0pt}{}{\parindent}{\bf}{}{\ccwd}{\thmname{#1}\thmnumber{ #2}\thmnote{#3}}
\theoremstyle{thmstyle}
\newtheorem{dingli}{定理}[section]
\newtheorem{jiashe}{假设}


%\renewcommand{\headrule}{\vbox to 10pt{\hbox to\headwidth{\hrulefill}\vss}}
\renewcommand{\headrulewidth}{0.75pt}
\renewcommand{\title}[1]{\gdef\art@title{#1}}
\renewcommand{\contentsname}{\bf\zihao{3}\centerline{目\quad 录}\vspace{-10pt}}
\renewcommand{\abstractname}{\rm\heiti\zihao{3} 摘\quad 要}
\renewcommand\refname{}
\renewcommand\figurename{图}
\renewcommand\tablename{表}
\renewcommand{\thefigure}{\thesection-\arabic{figure}}
\renewcommand{\thetable}{\thesection-\arabic{table}}
\renewcommand{\theequation}{\thesection-\arabic{equation}}
\renewcommand{\captionlabelfont}{\songti}
\renewcommand{\captiontextfont}{\songti}


\newcommand{\class}[1]{\gdef\@class{#1}}
\newcommand{\blankpage}{\vfill\clearpage\phantom{s}\thispagestyle{empty}}
\newcommand{\fancystyle}[3]{
    \pagestyle{fancy}{
        \fancyhf{}
        \fancyhead[CO]{\zihao{-5}#1}
        \fancyhead[CE]{\zihao{-5}#2}
        \fancyfoot[C]{\zihao{-5}#3\thepage}}}
\newcommand{\upcite}[1]{\!\textsuperscript{\cite{#1}}}
\newcommand{\sect}[1]{\vfill\clearpage\setcounter{figure}{0}\setcounter{table}{0}\setcounter{equation}{0}\section{#1}}
\newcommand{\figstyle}{\centering\vspace{-0.1cm}\setlength{\abovecaptionskip}{0.2cm}\setlength{\belowcaptionskip}{-0.5cm}}
\newcommand{\equstyle}{\setlength\abovedisplayskip{6pt}\setlength\belowdisplayskip{-6pt}}
\newcommand{\head}[1]{#1}
\newcommand{\cnabstract}[1]{\vfill\clearpage
    \fancystyle{摘要}{ABSTRACT}{\zihao{5}}
    \pagenumbering{Roman}
    \section*{\bf\zihao{3}摘\;\;要}#1}
\newcommand{\cnkeyword}[1]{\vskip 1.5em{\bf 关键词:}#1}
\newcommand{\enabstract}[1]{\vfill\clearpage
    \section*{\bf\zihao{3}ABSTRACT}#1}
\newcommand{\enkeyword}[1]{\vskip 1.5em{\bf Keywords: }#1}
\newcommand{\makecontents}{\vfill\clearpage
    \fancystyle{目录}{目录}{}
    \pagenumbering{roman}
    \begin{spacing}{1.5}\tableofcontents\end{spacing}}
\newcommand{\bodyfancystyle}{   %正文样式
    \fancystyle{\oddhead}{\evenhead}{}
    \pagenumbering{arabic}\setcounter{page}{1}}
\newcommand{\appendfancystyle}{  %附录样式
    \fancystyle{附录}{附录}{}}
\newcommand{\oddhead}{\art@title}
\newcommand{\evenhead}{\@class}
%\newcommand{\ziju}[1]{\renewcommand{\CJKglue}{\hskip #1pt}}

%\def\@cite#1#2{\textsuperscript{[{#1\if@tempswa,#2\fi}]}}
\captionsetup{font=small,labelsep=quad}%labelfont=bf,textfont=bf}
%\titleformat{\section}{\bfseries\Large}{\thesection}{1em}{}%标题靠左
\titleformat{\section}{\centering\bf\zihao{3}}{第\thesection 章}{0.8em}{}
\titleformat{\subsection}{\bf\zihao{4}}{{\thesection.\arabic{subsection}}}{0.6em}{}
\titleformat{\subsubsection}{\bf\zihao{-4}}{{\thesubsection.\arabic{subsubsection}}}{0.4em}{}
\DeclareFixedFont{\myfont}{OT1}{ptm}{m}{n}{16pt}


\titlecontents{section}[0em]{\zihao{4}}{第\thecontentslabel 章\hspace{0.8em}}{}
{\hspace{.5em}\titlerule*[4pt]{$\cdot$}{\contentspage}}
\titlecontents{subsection}[1em]{\vspace{0.1\baselineskip}\zihao{-4}}{\thecontentslabel \hspace{0.6em}} {}
{\hspace{.5em}\titlerule*[4pt]{$\cdot$}\contentspage}
\titlecontents{subsubsection}[2em]{\vspace{0.1\baselineskip}\zihao{5}}{\thecontentslabel \hspace{0.4em} }{}
{\hspace{.5em}\titlerule*[4pt]{$\cdot$}\contentspage}

%间距设置
%\titleformat{\chapter}[display]{ormalfont\huge\bfseries\center}{\chaptertitlename\ \thechapter}{10pt}{\Huge}
%\titleformat{\section}{ormalfont\Large\bfseries}{\thesection}{1em}{}
%\titleformat{\subsection}{ormalfont\large\bfseries}{\thesubsection}{1em}{}
%\titleformat{\subsubsection}{ormalfontormalsize\bfseries}{\thesubsubsection}{1em}{}
%\titleformat{\paragraph}[runin]{ormalfontormalsize\bfseries}{\theparagraph}{1em}{}
%\titleformat{\subparagraph}[runin]{ormalfontormalsize\bfseries}{\thesubparagraph}{1em}{}

%\titlespacing*{\chapter} {0pt}{50pt}{40pt}
%\titlespacing*{\section} {0pt}{3.5ex plus 1ex minus .2ex}{2.3ex plus .2ex}
%\titlespacing*{\subsection} {0pt}{3.25ex plus 1ex minus .2ex}{1.5ex plus .2ex}
%\titlespacing*{\subsubsection}{0pt}{3.25ex plus 1ex minus .2ex}{1.5ex plus .2ex}
%\titlespacing*{\paragraph} {0pt}{3.25ex plus 1ex minus .2ex}{1em}
%\titlespacing*{\subparagraph} {\parindent}{3.25ex plus 1ex minus .2ex}{1em}


%\showthe\baselineskip
%\renewcommand{\baselinestretch}{1}
%\setlength{\baselineskip}{30pt}
%\setlength{\lineskip}{0pt}
\setlength{\parindent}{2em}
\setlength{\parskip}{0pt}
\setlength{\topskip}{16pt}

%\setlength{\abovecaptionskip}{0.2cm}%图片标题与图片距离
%\setlength{\belowcaptionskip}{-0.4cm}%图片标题与下文距离

\titlespacing*{\section} {0pt}{*0}{*0.8}
\titlespacing*{\subsection} {0pt}{*0}{*0}
\titlespacing*{\subsubsection} {0pt}{*0}{*0}
%\titlespacing*{\paragraph} {0pt}{3.25ex plus 1ex minus .2ex}{-1em}
%\titlespacing*{\subparagraph} {0pt}{3.25ex plus 1ex minus .2ex}{-1em}


%代码设置
\RequirePackage{listings}
\RequirePackage{xcolor}
\definecolor{dkgreen}{rgb}{0,0.4,0}
\definecolor{gray}{rgb}{0.5,0.5,0.5}
\definecolor{mauve}{rgb}{0.58,0,0.82}
\newcommand{\esc}[1]{{\color{dkgreen}\kaishu\fontsize{11.25pt}\baselineskip#1\,}}
\lstset{
    backgroundcolor=\color{white},
    rulecolor=\color{gray!35},
    basicstyle=\ttfamily\color{black},
	keywordstyle=\color{blue}\bfseries,
	commentstyle=\color{dkgreen},
	stringstyle=\color{mauve}\ttfamily,
	numbers=left,  %显示行号
    numberstyle=\tiny\color{gray},
	%frame=tb,
    framerule=1pt,
	aboveskip=3mm,
	belowskip=3mm,
    showspaces=false,
	showstringspaces=false,
	columns=flexible,
	breaklines=true,
	breakatwhitespace=false,
    extendedchars=false,
    captionpos=b,
    tabsize=4,
}

\usepackage{hyperref}
\hypersetup{
    colorlinks=true,
    %bookmarks=true,
    %bookmarksopen=false,
    %bookmarksnumbered=false,
    linkcolor=black,
    filecolor=black,
    urlcolor=blue,
    citecolor=black,
}

%------------------------------------------------------------------------

\title{三相交流电机变频调速控制器的设计}
\class{DSP\,原理及应用}

%------------------------------------------------------------------------
%正文部分
\begin{document}

    \begin{titlepage}
        \vspace*{-2.75em}
        {\hskip 10.25cm\bf 编\hspace{0.4em} 号 \qquad 123456}

		\vspace*{1.9cm}
        \centering
		{\fontsize{50pt}\baselineskip\stxingkai 大学名称}\\
		\vspace*{2.45cm}
		{\fontsize{32pt}\baselineskip\bf{\ziju{0.16}本科生毕业设计(\,论文\,)}}
		 \vskip 2.3cm
        {\zihao{2}
		 \makebox[20mm]{\bf 题目:}
		 \underline{\makebox[116mm][c]{\sf 本科毕业设计(论文)题目}}\\
         \vskip 0.4cm
         \makebox[20mm]{}
         \underline{\makebox[116mm][c]{\sf  }}
         \vskip 1.95cm
         \zihao{-2}
         \underline{\makebox[60mm][c]{\sf  }}
         \makebox[13mm]{学\;院}
         \underline{\makebox[40mm][c]{\sf  }}
         \makebox[13mm]{专\;业}\\
         \vskip 1.85cm
         \makebox[28mm]{学\qquad 号}
		 \underline{\makebox[70mm][l]{\sf\qquad\quad 123456}}\hspace*{0.9em}	
		 \vskip 0.3cm
		 \makebox[28mm]{学生姓名}
		 \underline{\makebox[70mm][l]{\sf\qquad\quad 张\quad 三}}\hspace*{0.9em}
		 \vskip 0.3cm
		 \makebox[28mm]{指导教师}
		 \underline{\makebox[70mm][l]{\sf\qquad\quad 李\quad 四\quad 教授}}\hspace*{0.9em}
         \vskip 0.3cm
         \makebox[28mm]{}
		 \underline{\makebox[70mm][l]{\sf\qquad\quad 王\quad 五\quad 讲师 }}\hspace*{0.9em}
		 \vskip 2.45cm
		 {\sf\zhdigits{\the\year}}年{\sf\zhnumber{\the\month}}月}		
	\end{titlepage}

    \blankpage

    \cnabstract{三相交流电机的变频调速,用数字信号处理器DSP}
    \cnkeyword{三相交流电机;变频调速}

    \enabstract{this is xelatex}
    \enkeyword{test; test}

    \makecontents
    \blankpage


    \sect{概述}
    \bodyfancystyle
    这是测试文字\upcite{intro},这是测试文字这是测试文字这是测试文字

        \subsection{第一小节}
        这是测试文字这是测试文字这是测试文字这是测试文字这是测试文字这是测试文字这是测试文字这是测试文字

        这是测试文字这是测试文字这是测试文字这是测试文字这是测试文字
            \subsubsection{第一小小节}
            这是测试文字这是测试文字这是测试文字这是测试文字这是测试文字,如图\,\ref{fig-t}\,所示.
                \begin{figure}[H]
                    \figstyle
            	    \includegraphics [width=0.1\textwidth]{figures/test.png}
            	    \caption{测试}\label{fig-t}
                \end{figure}

            这是测试文字这是测试文字这是测试文字这是测试文字这是测试文字这是测试文字
            \subsubsection{第二小小节}
            这是测试文字这是测试文字这是测试文字这是测试文字这是测试文字,如表\,\ref{tab-t}\,所示.
                \begin{table}[H]
                    \centering
                    \setlength{\abovecaptionskip}{0.1cm}
                    \setlength{\belowcaptionskip}{-0.1cm}
                    \caption{测试}\label{tab-t}
                    \begin{tabu}{|c|c|[1.2pt]ccc|}
                    \hline
                    \multicolumn{2}{|c|[1.2pt]}{\multirow{2}{*}{M}}
                    & \multicolumn{3}{c|}{A}\\
                    \cline{3-5}
                    \multicolumn{2}{|c|[1.2pt]}{}&a&b&c\\
                    \tabucline[1.2pt]{-}
                    \multirow{3}{*}{B}
                    &m&1&2&3\\
                    &n&4&5&6\\
                    &p&7&8&9\\
                    \hline
                    \end{tabu}
                    \vspace{-0.3cm}
                \end{table}

            这是测试文字这是测试文字这是测试文字这是测试文字这是测试文字
                \begin{table}[H]
                    \centering
                    \setlength{\abovecaptionskip}{0.1cm}
                    \setlength{\belowcaptionskip}{-0.1cm}
                    \caption{测试}\label{tab-m}
                    \begin{tabular}{ccc}
                        \toprule[1.2pt]
                        A&B&C\\
                        \midrule
                        a&b&c\\
                        \bottomrule[1.2pt]
                    \end{tabular}
                    \vspace{-0.3cm}
                \end{table}

    \sect{原理}
    这是测试文字
        \subsection{第二小节}
        %\footnote{这是测试文字}
        这是测试文字这是测试文字这是测试文字这是测试文字这是测试文字
        \begin{dingli}[(李雅普洛夫)]
          这是测试文字
        \end{dingli}

        \begin{dingli}
          这是测试文字
        \end{dingli}

            \begin{figure}[H]
                \figstyle
                \subfigure[test1]{
                    \begin{minipage}[t]{0.48\textwidth}
                        \centering
                	    \includegraphics [width=1in]{figures/test.png}
                        \label{subfig-a}
                    \end{minipage}}
                \subfigure[test2]{
                    \begin{minipage}[t]{0.48\textwidth}
                        \centering
                	    \includegraphics [width=1in]{figures/test.png}
                        \label{subfig-b}
                    \end{minipage}}
                \caption{test\protect\\ ab}
            \end{figure}

        这是测试文字这是测试文字这是测试文字这是测试文字这是测试文字
            \begin{figure}[H]
                \figstyle
                \begin{minipage}[t]{0.3\textwidth}
                    \centering
            	    \includegraphics [width=1in]{figures/test.png}
                    \caption{fig1}
                \end{minipage}
                \begin{minipage}[t]{0.3\textwidth}
                    \centering
            	    \includegraphics [width=1in]{figures/test.png}
                    \caption{fig2}
                \end{minipage}
            \end{figure}

            这是测试文字这是测试文字这是测试文字这是测试文字这是测试文字这是测试文字这是测试文字
            这是测试文字这是测试文字这是测试文字这是测试文字这是测试文字这是测试文字这是测试文字
            这是测试文字这是测试文字这是测试文字这是测试文字

            \begin{wrapfigure}{r}{0.2\textwidth}
                \figstyle
                \vspace{-12pt}
                \includegraphics [width=0.18\textwidth]{figures/test.png}
                \caption{test}
            \end{wrapfigure}

            这是测试文字这是测试文字这是测试文字这是测试文字这是测试文字这是测试文字这是测试文字这是测试文字这是测试文字这是测试文字这是测试文字这是测试文字这是测试文字这是测试文字这是测试文字这是测试文字这是测试文字这是测试文字这是测试文字这是测试文字这是测试文字这是测试文字这是测试文字这是测试文字这是测试文字文字这是测试文字这是测试文字
            \subsubsection{第二小小节}
            这是测试文字这是测试文字这是测试文字这是测试文字这是测试文字
                {\equstyle
                \begin{align}
                    G(s)=\frac{1/0.368}{0.0144s+1}
                    \label{equ-gs}
                \end{align}}

            式(\ref{equ-gs})说明了文字这是测试文字这是测试文字这是测试文字这是测试文字这是测试文字

             \begin{align}
            {\Arrowvert x^{(1)}_{k+1}-x^{(1)}_k\Arrowvert}_\lambda\leqslant \;&
            \frac{1}{1-h_1(i)}\Arrowvert x^{(1)}_{k+1}(0)-x^{(1)}_k(0)\Arrowvert+\notag\\
            &\frac{h_2(i)}{1-h_1(i)}{\Arrowvert
            	e^{(1)}_k\Arrowvert}_\lambda+\rho(Q).
            \end{align}

    \sect{硬件设计}
    这是测试文字
        \subsection{系统组成框图}

    \sect{软件设计}
这是测试文字

    \sect{调试}
这是测试文字

    \sect{总结}
这是测试文字

    \setcounter{secnumdepth}{-2}

    \vfill
    \clearpage
    \section{参考文献}
    \begin{thebibliography}{99}
        \vspace{-2.35em}
        %\addcontentsline{toc}{section}{参考文献}
        %\setlength{\bibsep}{0.5ex}
        \addtolength{\itemsep}{-0.7em}

        \bibitem{intro}张卫平.\;\;开关变换器的建模与控制[M].\;\;北京:中国电力出版社,2006-9:15-88.

    \end{thebibliography}


    \vfill
    \clearpage
    \section{致\texorpdfstring{\quad}{}谢}



    \vfill
    \clearpage
    \appendfancystyle
    \begin{appendix}
    \section{附录A:\texorpdfstring{\;}{}程序代码}

        \begin{lstlisting}[language=C,escapeinside=``]
#include <stdio.h>

//test code
int main() {
    return 0;
}

        \end{lstlisting}
    \end{appendix}

\end{document}
