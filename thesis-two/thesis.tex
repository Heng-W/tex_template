\documentclass[UTF8,zihao=5,twoside]{ctexart}
\usepackage{fancyhdr}%页眉页脚
\usepackage{graphicx}%图片
\usepackage{amsmath,amsfonts,amssymb}%数学公式,符号
\usepackage{array}%表格、数组
\usepackage{cases}%左半边括号
\usepackage{listings}%附录
%\usepackage{lastpage}
\usepackage{subfigure}
\usepackage{helvet}
\usepackage{setspace}
\usepackage{upgreek}
\usepackage{multirow}
\usepackage{multicol}
\usepackage{appendix}
\usepackage{titlesec}
\usepackage{titletoc}
\usepackage{cite}
\usepackage{tabu}
\usepackage{booktabs}
\usepackage{bm}
\usepackage{xeCJK}
\usepackage{fontspec,xunicode,xltxtra}
\usepackage{latexsym}
\usepackage{caption}
\usepackage{wrapfig}
%\usepackage{mathtime}
%\usepackage{subcaption}
\usepackage{float}
\usepackage{verbatim}
%\usepackage{mathspec}
\usepackage[T1]{fontenc}
\usepackage{mathptmx}
%\usepackage{newtxmath}
\usepackage{ifthen}
\usepackage{indentfirst}
\usepackage{zhnumber}
\usepackage[a4paper,left=2cm,right=2cm,top=2cm,bottom=2cm,headheight=39pt,headsep=8pt]{geometry}


\setCJKmainfont[AutoFakeBold=2.5]{SimSun}%BoldFont=SimHei,ItalicFont=KaiTi
%\setCJKmonofont{FangSong}
%\setCJKsansfont[SimHei}
\setmainfont{Times New Roman}
%\setallmathfont{Times New Roman}
\linespread{1.25}


\newboolean{ifcntitle}
\newboolean{ifcnauthor}
\newboolean{ifcnabstract}

\newboolean{ifentitle}
\newboolean{ifenauthor}
\newboolean{ifenabstract}

\renewcommand{\headrule}{{\rule[0.6\baselineskip]{\headwidth}{0.75pt}}\vss}
%\renewcommand{\headrule}{\vbox to 10pt{\hbox to\headwidth{\hrulefill}\vss}}
%\renewcommand{\headrulewidth}{0.75pt}
\renewcommand{\maketitle}{
    \ifthenelse{\boolean{ifcntitle}}{\vspace*{-0.8em}\makecntitle}{}
    \ifthenelse{\boolean{ifcnauthor}}{\vskip 0.5em\cnauthorDef}{}
    \ifthenelse{\boolean{ifcnabstract}}{\vskip 0.8em\cnabstractDef\vskip 0.8em\cnkeywordDef}{}
    \ifthenelse{\boolean{ifentitle}}{\vskip 1em\makeentitle}{}
    \ifthenelse{\boolean{ifenauthor}}{\vskip 0.5em\enauthorDef}{}
    \ifthenelse{\boolean{ifenabstract}}{\vskip 0.8em\enabstractDef\vskip 0.8em\enkeywordDef}{}
    \vskip 2em}
%\renewcommand{\contentsname}{\bf\zihao{3}\centerline{目\quad 录}\vspace{-10pt}}
%\renewcommand{\abstractname}{\rm\heiti\zihao{3} 摘\quad 要}
%\renewcommand\thefigure{图}
%\renewcommand\tablename{表}
\renewcommand\refname{}
\renewcommand{\thefigure}{\arabic{figure}}
\renewcommand{\thetable}{\arabic{table}}
\renewcommand{\theequation}{\arabic{equation}}
\renewcommand{\captionlabelfont}{\bf}
\renewcommand{\captiontextfont}{\heiti}


\newcommand{\cntitle}[1]{\gdef\cntitleDef{#1}}
\newcommand{\cnauthor}[2]{\setboolean{ifcnauthor}{true}\gdef\cnauthorDef{
    \begin{spacing}{1.5}{\centering\zihao{-4}{\zihao{4}\kaishu{#1}}\par #2\vskip -0.4em}\end{spacing}}}
\newcommand{\entitle}[1]{\setboolean{ifentitle}{true}\gdef\entitleDef{#1}}
\newcommand{\enauthor}[2]{\setboolean{ifenauthor}{true}\gdef\enauthorDef{
    \begin{spacing}{1.5}{\centering\zihao{-4}{\it{#1}}\par #2\vskip -0.4em}\end{spacing}}}
\newcommand{\cnclass}[1]{\gdef\cnclassDef{#1}}
\newcommand{\enclass}[1]{\gdef\enclassDef{#1}}

\newcommand{\makecntitle}{{\centering\zihao{-2}\bf\cntitleDef\par}}
\newcommand{\makeentitle}{{\centering\zihao{-2}\bf\entitleDef\par}}
\newcommand{\blankpage}{\vfill\clearpage\phantom{s}\thispagestyle{empty}}
\newcommand{\fancystyle}[3]{
    \pagestyle{fancy}{
        \fancyhf{}
        \fancyhead[CO]{\kaishu\zihao{-4}#1}
        \fancyhead[CE]{\kaishu\zihao{-4}#2}
        \fancyfoot[C]{\zihao{-4}#3\thepage}}}
\newcommand{\upcite}[1]{\!\textsuperscript{\cite{#1}}}
\newcommand{\sect}[1]{\setcounter{figure}{0}\setcounter{table}{0}\setcounter{equation}{0}\section{#1}}
\newcommand{\figstyle}{\centering\vspace{-0.1cm}\setlength{\abovecaptionskip}{0.3cm}\setlength{\belowcaptionskip}{-0.5cm}}
\newcommand{\equstyle}{\setlength\abovedisplayskip{6pt}\setlength\belowdisplayskip{-6pt}}
\newcommand{\cnabstract}[1]{\setboolean{ifcnabstract}{true}\gdef\cnabstractDef{\noindent{\bf\zihao{-4}摘要:}#1}}
\newcommand{\cnkeyword}[1]{\gdef\cnkeywordDef{\noindent{\bf 关键词:}#1}}
\newcommand{\enabstract}[1]{\setboolean{ifenabstract}{true}\gdef\enabstractDef{\noindent{\bf\zihao{-4}Abstract: }#1}}
\newcommand{\enkeyword}[1]{\gdef\enkeywordDef{\noindent{\bf Keywords: }#1}}
\newcommand{\makecontents}{\vfill\clearpage
    \begin{spacing}{1.5}\tableofcontents\end{spacing}}
\newcommand{\bodyfancystyle}{   %正文样式
    \fancystyle{\ifthenelse{\thepage=1}{\cnclassDef\\\it\enclassDef}{\oddhead}}{\evenhead}{}
    \pagenumbering{arabic}\setcounter{page}{1}}
\newcommand{\oddhead}[1]{\cntitleDef}
\newcommand{\evenhead}[1]{\cnclassDef}


%\def\@cite#1#2{\textsuperscript{[{#1\if@tempswa,#2\fi}]}}
\captionsetup{labelsep=quad}%labelfont=bf,textfont=bf}
%\titleformat{\section}{\bfseries\Large}{\thesection}{1em}{}%标题靠左
\titleformat{\section}{\bf\zihao{3}}{{\thesection}}{0.8em}{}
\titleformat{\subsection}{\bf\zihao{4}}{{\thesection.\arabic{subsection}}}{0.6em}{}
\titleformat{\subsubsection}{\bf\zihao{-4}}{{\thesubsection.\arabic{subsubsection}}}{0.4em}{}
\DeclareFixedFont{\myfont}{OT1}{ptm}{m}{n}{17pt}


\titlecontents{section}[0em]{\zihao{4}}{\thecontentslabel\hspace{0.8em}}{}
{\hspace{.5em}\titlerule*[4pt]{$\cdot$}{\contentspage}}
\titlecontents{subsection}[1em]{\vspace{0.1\baselineskip}\zihao{-4}}{\thecontentslabel \hspace{0.6em}} {}
{\hspace{.5em}\titlerule*[4pt]{$\cdot$}\contentspage}
\titlecontents{subsubsection}[2em]{\vspace{0.1\baselineskip}\zihao{5}}{\thecontentslabel \hspace{0.4em} }{}
{\hspace{.5em}\titlerule*[4pt]{$\cdot$}\contentspage}

%间距设置
%\titleformat{\chapter}[display]{ormalfont\huge\bfseries\center}{\chaptertitlename\ \thechapter}{10pt}{\Huge}
%\titleformat{\section}{ormalfont\Large\bfseries}{\thesection}{1em}{}
%\titleformat{\subsection}{ormalfont\large\bfseries}{\thesubsection}{1em}{}
%\titleformat{\subsubsection}{ormalfontormalsize\bfseries}{\thesubsubsection}{1em}{}
%\titleformat{\paragraph}[runin]{ormalfontormalsize\bfseries}{\theparagraph}{1em}{}
%\titleformat{\subparagraph}[runin]{ormalfontormalsize\bfseries}{\thesubparagraph}{1em}{}

%\titlespacing*{\chapter} {0pt}{50pt}{40pt}
%\titlespacing*{\section} {0pt}{3.5ex plus 1ex minus .2ex}{2.3ex plus .2ex}
%\titlespacing*{\subsection} {0pt}{3.25ex plus 1ex minus .2ex}{1.5ex plus .2ex}
%\titlespacing*{\subsubsection}{0pt}{3.25ex plus 1ex minus .2ex}{1.5ex plus .2ex}
%\titlespacing*{\paragraph} {0pt}{3.25ex plus 1ex minus .2ex}{1em}
%\titlespacing*{\subparagraph} {\parindent}{3.25ex plus 1ex minus .2ex}{1em}


%\showthe\baselineskip
%\renewcommand{\baselinestretch}{1}
%\setlength{\baselineskip}{30pt}
%\setlength{\lineskip}{0pt}
\setlength{\parindent}{2em}
\setlength{\parskip}{0pt}
\setlength{\topskip}{0pt}

%\setlength{\abovecaptionskip}{0.2cm}%图片标题与图片距离
%\setlength{\belowcaptionskip}{-0.4cm}%图片标题与下文距离

\titlespacing*{\section} {0pt}{*0.1}{*0.2}
\titlespacing*{\subsection} {0pt}{*0}{*0}
\titlespacing*{\subsubsection} {0pt}{*0}{*0}
%\titlespacing*{\paragraph} {0pt}{3.25ex plus 1ex minus .2ex}{-1em}
%\titlespacing*{\subparagraph} {0pt}{3.25ex plus 1ex minus .2ex}{-1em}


%代码设置
\RequirePackage{listings}
\RequirePackage{xcolor}
\definecolor{dkgreen}{rgb}{0,0.6,0}
\definecolor{gray}{rgb}{0.5,0.5,0.5}
\definecolor{mauve}{rgb}{0.58,0,0.82}
\newcommand{\esc}[1]{{\color{dkgreen}\kaishu\fontsize{11.25pt}\baselineskip#1\,}}
\lstset{
    backgroundcolor=\color{white},
    rulecolor=\color{gray!35},
    basicstyle=\ttfamily\color{black},
	keywordstyle=\color{blue}\bfseries,
	commentstyle=\color{dkgreen},
	stringstyle=\color{mauve}\ttfamily,
	numbers=left,  %显示行号
    numberstyle=\tiny\color{gray},
	%frame=tb,
    framerule=1pt,
	aboveskip=3mm,
	belowskip=3mm,
    showspaces=false,
	showstringspaces=false,
	columns=flexible,
	breaklines=true,
	breakatwhitespace=false,
    extendedchars=false,
    captionpos=b,
    tabsize=4,
}

\usepackage{hyperref}
\hypersetup{
    colorlinks=true,
    %bookmarks=true,
    %bookmarksopen=false,
    %bookmarksnumbered=false,
    linkcolor=black,
    filecolor=black,
    urlcolor=blue,
    citecolor=black,
}

%------------------------------------------------------------------------

\cntitle{机器人}
\cnclass{工业机器人技术}
\enclass{Industrial Robotics}

\setboolean{ifcntitle}{true}
\cnauthor{张~~~~三}{(大学名,学院名,省名~城市名~邮政编码)}

\cnabstract{这是测试文字这是测试文字这是测试文字这是测试文字这是测试文字这是测试文字这是测试文字这是测试文字这是测试文字}
\cnkeyword{测试文字;测试文字}

\entitle{Robot}
\enauthor{ZHANG San}{(University, College, City~postcode, China)}

\enabstract{this is xelatex}
\enkeyword{test; test}

%------------------------------------------------------------------------
%正文部分
\begin{document}

    \clearpage
    \bodyfancystyle

    \maketitle


    \begin{multicols}{2}
    \setcounter{section}{-1}
    \section{引言}
        这是测试文字这是测试文字

    \section{概括}
    这是测试文字\upcite{intro},这是测试文字这是测试文字这是测试文字
        \subsection{第一小节}
        这是测试文字这是测试文字
            \subsubsection{第一小小节}
            这是测试文字这是测试文字这是测试文字这是测试文字这是测试文字,如图\,\ref{fig-t}\,所示.
                \begin{figure}[H]
                    \figstyle
            	    \includegraphics [width=0.1\textwidth]{figures/test.png}
            	    \caption{测试}\label{fig-t}
                \end{figure}

            这是测试文字这是测试文字这是测试文字这是测试文字这是测试文字这是测试文字

            这是测试文字这是测试文字
            \subsubsection{第二小小节}
            这是测试文字这是测试文字这是测试文字这是测试文字这是测试文字,如表\,\ref{tab-t}\,所示.
                \begin{table}[H]
                    \centering
                    \setlength{\abovecaptionskip}{0.1cm}
                    \setlength{\belowcaptionskip}{-0.1cm}
                    \caption{测试}\label{tab-t}
                    \begin{tabu}{|c|c|[1.2pt]ccc|}
                    \hline
                    \multicolumn{2}{|c|[1.2pt]}{\multirow{2}{*}{M}}
                    & \multicolumn{3}{c|}{A}\\
                    \cline{3-5}
                    \multicolumn{2}{|c|[1.2pt]}{}&a&b&c\\
                    \tabucline[1.2pt]{-}
                    \multirow{3}{*}{B}
                    &m&1&2&3\\
                    &n&4&5&6\\
                    &p&7&8&9\\
                    \hline
                    \end{tabu}
                    \vspace{-0.3cm}
                \end{table}

            这是测试文字这是测试文字这是测试文字这是测试文字这是测试文字
                \begin{table}[H]
                    \centering
                    \setlength{\abovecaptionskip}{0.1cm}
                    \setlength{\belowcaptionskip}{-0.1cm}
                    \caption{测试}\label{tab-m}
                    \begin{tabular}{ccc}
                        \toprule[1.2pt]
                        A&B&C\\
                        \midrule
                        a&b&c\\
                        \bottomrule[1.2pt]
                    \end{tabular}
                    \vspace{-0.3cm}
                \end{table}

    \section{原理}
    这是测试文字
        \subsection{第二小节}
        %\footnote{这是测试文字}
        这是测试文字这是测试文字这是测试文字这是测试文字这是测试文字


        这是测试文字这是测试文字这是测试文字这是测试文字这是测试文字



            \subsubsection{第二小小节}
            这是测试文字这是测试文字这是测试文字这是测试文字这是测试文字
                {\equstyle
                \begin{align}
                    G(s)=\frac{1/0.368}{0.0144s+1}
                    \label{equ-gs}
                \end{align}}

            式(\ref{equ-gs})说明了文字这是测试文字这是测试文字这是测试文字这是测试文字这是测试文字

    \section{硬件设计}
    这是测试文字
        \subsection{系统组成框图}
            \begin{align}
            {\Arrowvert x^{(1)}_{k+1}-x^{(1)}_k\Arrowvert}_\lambda\leqslant \;&
            \frac{1}{1-h_1(i)}\Arrowvert x^{(1)}_{k+1}(0)-x^{(1)}_k(0)\Arrowvert+\notag\\
            &\frac{h_2(i)}{1-h_1(i)}{\Arrowvert
            	e^{(1)}_k\Arrowvert}_\lambda+\rho(Q).
            \end{align}
    \section{软件设计}
这是测试文字

    \section{调试}

    \section{总结}
这是测试文字

    \setcounter{secnumdepth}{-2}


    \section{\texorpdfstring{\zihao{4}}{}参考文献(References)}
    \begin{thebibliography}{99}
        \vspace{-1.8em}
        %\addcontentsline{toc}{section}{参考文献}
        %\setlength{\bibsep}{0.5ex}
        \addtolength{\itemsep}{-0.7em}

        \bibitem{intro}张卫平.\;\;开关变换器的建模与控制[M].\;\;北京:中国电力出版社,2006-9:15-88.

    \end{thebibliography}

    \end{multicols}


\begin{comment}
    \begin{appendix}
    \section{附录A:\texorpdfstring{\;}{}程序代码}

        \begin{lstlisting}[language=C,escapeinside=``]
#include "stdio.h"

int main() {
    return 0;
}

        \end{lstlisting}
    \end{appendix}
\end{comment}

\end{document}
